% Options for packages loaded elsewhere
\PassOptionsToPackage{unicode}{hyperref}
\PassOptionsToPackage{hyphens}{url}
%
\documentclass[
]{article}
\usepackage{amsmath,amssymb}
\usepackage{iftex}
\ifPDFTeX
  \usepackage[T1]{fontenc}
  \usepackage[utf8]{inputenc}
  \usepackage{textcomp} % provide euro and other symbols
\else % if luatex or xetex
  \usepackage{unicode-math} % this also loads fontspec
  \defaultfontfeatures{Scale=MatchLowercase}
  \defaultfontfeatures[\rmfamily]{Ligatures=TeX,Scale=1}
\fi
\usepackage{lmodern}
\ifPDFTeX\else
  % xetex/luatex font selection
\fi
% Use upquote if available, for straight quotes in verbatim environments
\IfFileExists{upquote.sty}{\usepackage{upquote}}{}
\IfFileExists{microtype.sty}{% use microtype if available
  \usepackage[]{microtype}
  \UseMicrotypeSet[protrusion]{basicmath} % disable protrusion for tt fonts
}{}
\makeatletter
\@ifundefined{KOMAClassName}{% if non-KOMA class
  \IfFileExists{parskip.sty}{%
    \usepackage{parskip}
  }{% else
    \setlength{\parindent}{0pt}
    \setlength{\parskip}{6pt plus 2pt minus 1pt}}
}{% if KOMA class
  \KOMAoptions{parskip=half}}
\makeatother
\usepackage{xcolor}
\usepackage[margin=1in]{geometry}
\usepackage{color}
\usepackage{fancyvrb}
\newcommand{\VerbBar}{|}
\newcommand{\VERB}{\Verb[commandchars=\\\{\}]}
\DefineVerbatimEnvironment{Highlighting}{Verbatim}{commandchars=\\\{\}}
% Add ',fontsize=\small' for more characters per line
\usepackage{framed}
\definecolor{shadecolor}{RGB}{248,248,248}
\newenvironment{Shaded}{\begin{snugshade}}{\end{snugshade}}
\newcommand{\AlertTok}[1]{\textcolor[rgb]{0.94,0.16,0.16}{#1}}
\newcommand{\AnnotationTok}[1]{\textcolor[rgb]{0.56,0.35,0.01}{\textbf{\textit{#1}}}}
\newcommand{\AttributeTok}[1]{\textcolor[rgb]{0.13,0.29,0.53}{#1}}
\newcommand{\BaseNTok}[1]{\textcolor[rgb]{0.00,0.00,0.81}{#1}}
\newcommand{\BuiltInTok}[1]{#1}
\newcommand{\CharTok}[1]{\textcolor[rgb]{0.31,0.60,0.02}{#1}}
\newcommand{\CommentTok}[1]{\textcolor[rgb]{0.56,0.35,0.01}{\textit{#1}}}
\newcommand{\CommentVarTok}[1]{\textcolor[rgb]{0.56,0.35,0.01}{\textbf{\textit{#1}}}}
\newcommand{\ConstantTok}[1]{\textcolor[rgb]{0.56,0.35,0.01}{#1}}
\newcommand{\ControlFlowTok}[1]{\textcolor[rgb]{0.13,0.29,0.53}{\textbf{#1}}}
\newcommand{\DataTypeTok}[1]{\textcolor[rgb]{0.13,0.29,0.53}{#1}}
\newcommand{\DecValTok}[1]{\textcolor[rgb]{0.00,0.00,0.81}{#1}}
\newcommand{\DocumentationTok}[1]{\textcolor[rgb]{0.56,0.35,0.01}{\textbf{\textit{#1}}}}
\newcommand{\ErrorTok}[1]{\textcolor[rgb]{0.64,0.00,0.00}{\textbf{#1}}}
\newcommand{\ExtensionTok}[1]{#1}
\newcommand{\FloatTok}[1]{\textcolor[rgb]{0.00,0.00,0.81}{#1}}
\newcommand{\FunctionTok}[1]{\textcolor[rgb]{0.13,0.29,0.53}{\textbf{#1}}}
\newcommand{\ImportTok}[1]{#1}
\newcommand{\InformationTok}[1]{\textcolor[rgb]{0.56,0.35,0.01}{\textbf{\textit{#1}}}}
\newcommand{\KeywordTok}[1]{\textcolor[rgb]{0.13,0.29,0.53}{\textbf{#1}}}
\newcommand{\NormalTok}[1]{#1}
\newcommand{\OperatorTok}[1]{\textcolor[rgb]{0.81,0.36,0.00}{\textbf{#1}}}
\newcommand{\OtherTok}[1]{\textcolor[rgb]{0.56,0.35,0.01}{#1}}
\newcommand{\PreprocessorTok}[1]{\textcolor[rgb]{0.56,0.35,0.01}{\textit{#1}}}
\newcommand{\RegionMarkerTok}[1]{#1}
\newcommand{\SpecialCharTok}[1]{\textcolor[rgb]{0.81,0.36,0.00}{\textbf{#1}}}
\newcommand{\SpecialStringTok}[1]{\textcolor[rgb]{0.31,0.60,0.02}{#1}}
\newcommand{\StringTok}[1]{\textcolor[rgb]{0.31,0.60,0.02}{#1}}
\newcommand{\VariableTok}[1]{\textcolor[rgb]{0.00,0.00,0.00}{#1}}
\newcommand{\VerbatimStringTok}[1]{\textcolor[rgb]{0.31,0.60,0.02}{#1}}
\newcommand{\WarningTok}[1]{\textcolor[rgb]{0.56,0.35,0.01}{\textbf{\textit{#1}}}}
\usepackage{graphicx}
\makeatletter
\def\maxwidth{\ifdim\Gin@nat@width>\linewidth\linewidth\else\Gin@nat@width\fi}
\def\maxheight{\ifdim\Gin@nat@height>\textheight\textheight\else\Gin@nat@height\fi}
\makeatother
% Scale images if necessary, so that they will not overflow the page
% margins by default, and it is still possible to overwrite the defaults
% using explicit options in \includegraphics[width, height, ...]{}
\setkeys{Gin}{width=\maxwidth,height=\maxheight,keepaspectratio}
% Set default figure placement to htbp
\makeatletter
\def\fps@figure{htbp}
\makeatother
\setlength{\emergencystretch}{3em} % prevent overfull lines
\providecommand{\tightlist}{%
  \setlength{\itemsep}{0pt}\setlength{\parskip}{0pt}}
\setcounter{secnumdepth}{-\maxdimen} % remove section numbering
\ifLuaTeX
  \usepackage{selnolig}  % disable illegal ligatures
\fi
\usepackage{bookmark}
\IfFileExists{xurl.sty}{\usepackage{xurl}}{} % add URL line breaks if available
\urlstyle{same}
\hypersetup{
  pdftitle={TP 1: Regresión ordinal},
  pdfauthor={Nicolás Celie, Martín Peralta, Nicolás Ian Rozenberg},
  hidelinks,
  pdfcreator={LaTeX via pandoc}}

\title{TP 1: Regresión ordinal}
\author{Nicolás Celie, Martín Peralta, Nicolás Ian Rozenberg}
\date{2025-06-07}

\begin{document}
\maketitle

\begin{Shaded}
\begin{Highlighting}[]
\FunctionTok{library}\NormalTok{(tidyverse)}
\end{Highlighting}
\end{Shaded}

\begin{verbatim}
## Warning: package 'tidyverse' was built under R version 4.4.3
\end{verbatim}

\begin{verbatim}
## Warning: package 'ggplot2' was built under R version 4.4.1
\end{verbatim}

\begin{verbatim}
## Warning: package 'tibble' was built under R version 4.4.1
\end{verbatim}

\begin{verbatim}
## Warning: package 'tidyr' was built under R version 4.4.1
\end{verbatim}

\begin{verbatim}
## Warning: package 'readr' was built under R version 4.4.1
\end{verbatim}

\begin{verbatim}
## Warning: package 'dplyr' was built under R version 4.4.1
\end{verbatim}

\begin{verbatim}
## Warning: package 'forcats' was built under R version 4.4.3
\end{verbatim}

\begin{verbatim}
## Warning: package 'lubridate' was built under R version 4.4.1
\end{verbatim}

\begin{verbatim}
## -- Attaching core tidyverse packages ------------------------ tidyverse 2.0.0 --
## v dplyr     1.1.4     v readr     2.1.5
## v forcats   1.0.0     v stringr   1.5.1
## v ggplot2   3.5.1     v tibble    3.2.1
## v lubridate 1.9.3     v tidyr     1.3.1
## v purrr     1.0.2     
## -- Conflicts ------------------------------------------ tidyverse_conflicts() --
## x dplyr::filter() masks stats::filter()
## x dplyr::lag()    masks stats::lag()
## i Use the conflicted package (<http://conflicted.r-lib.org/>) to force all conflicts to become errors
\end{verbatim}

\begin{Shaded}
\begin{Highlighting}[]
\FunctionTok{library}\NormalTok{(MASS)}
\end{Highlighting}
\end{Shaded}

\begin{verbatim}
## 
## Adjuntando el paquete: 'MASS'
## 
## The following object is masked from 'package:dplyr':
## 
##     select
\end{verbatim}

\begin{Shaded}
\begin{Highlighting}[]
\FunctionTok{library}\NormalTok{(broom) }
\end{Highlighting}
\end{Shaded}

\begin{verbatim}
## Warning: package 'broom' was built under R version 4.4.3
\end{verbatim}

\begin{Shaded}
\begin{Highlighting}[]
\FunctionTok{library}\NormalTok{(janitor)   }
\end{Highlighting}
\end{Shaded}

\begin{verbatim}
## Warning: package 'janitor' was built under R version 4.4.3
\end{verbatim}

\begin{verbatim}
## 
## Adjuntando el paquete: 'janitor'
## 
## The following objects are masked from 'package:stats':
## 
##     chisq.test, fisher.test
\end{verbatim}

\begin{Shaded}
\begin{Highlighting}[]
\FunctionTok{library}\NormalTok{(rsample)}
\end{Highlighting}
\end{Shaded}

\begin{verbatim}
## Warning: package 'rsample' was built under R version 4.4.3
\end{verbatim}

\subsection{Ejercicio 1}\label{ejercicio-1}

Dado que al hacer el primer análisis exploratorio nos dimos cuenta que
había valores de edad completamente irrisorios, sacamos los outliers del
dataset para tener datos representativos. Luego de ello, realizamos la
separación en train-test dejando una proporción 80/20.

\begin{Shaded}
\begin{Highlighting}[]
\FunctionTok{set.seed}\NormalTok{(}\DecValTok{123}\NormalTok{)}

\NormalTok{df }\OtherTok{\textless{}{-}} \FunctionTok{read.csv2}\NormalTok{(}\StringTok{"data/data.csv"}\NormalTok{, }\AttributeTok{sep=}\StringTok{"}\SpecialCharTok{\textbackslash{}t}\StringTok{"}\NormalTok{) }\SpecialCharTok{|\textgreater{}} 
  \FunctionTok{as.data.frame}\NormalTok{()}

\CommentTok{\# Filtramos outliers}
\NormalTok{Q1 }\OtherTok{\textless{}{-}} \FunctionTok{quantile}\NormalTok{(df}\SpecialCharTok{$}\NormalTok{age, }\FloatTok{0.25}\NormalTok{, }\AttributeTok{na.rm =} \ConstantTok{TRUE}\NormalTok{)}
\NormalTok{Q3 }\OtherTok{\textless{}{-}} \FunctionTok{quantile}\NormalTok{(df}\SpecialCharTok{$}\NormalTok{age, }\FloatTok{0.75}\NormalTok{, }\AttributeTok{na.rm =} \ConstantTok{TRUE}\NormalTok{)}
\NormalTok{IQR }\OtherTok{\textless{}{-}}\NormalTok{ Q3 }\SpecialCharTok{{-}}\NormalTok{ Q1}

\NormalTok{limite\_inferior }\OtherTok{\textless{}{-}}\NormalTok{ Q1 }\SpecialCharTok{{-}} \FloatTok{1.5} \SpecialCharTok{*}\NormalTok{ IQR}
\NormalTok{limite\_superior }\OtherTok{\textless{}{-}}\NormalTok{ Q3 }\SpecialCharTok{+} \FloatTok{1.5} \SpecialCharTok{*}\NormalTok{ IQR}

\NormalTok{df }\OtherTok{\textless{}{-}}\NormalTok{ df  }\SpecialCharTok{|\textgreater{}}  
  \FunctionTok{filter}\NormalTok{(age}\SpecialCharTok{\textless{}}\NormalTok{limite\_superior,}
\NormalTok{         age}\SpecialCharTok{\textgreater{}}\NormalTok{limite\_inferior)}

\CommentTok{\# Split Train {-} Test}
\NormalTok{split }\OtherTok{\textless{}{-}} \FunctionTok{initial\_split}\NormalTok{(df, }\AttributeTok{prop =} \FloatTok{0.8}\NormalTok{)}

\NormalTok{df\_train }\OtherTok{\textless{}{-}} \FunctionTok{training}\NormalTok{(split)}
\NormalTok{df\_test  }\OtherTok{\textless{}{-}} \FunctionTok{testing}\NormalTok{(split)}
\end{Highlighting}
\end{Shaded}

\subsection{Ejercicio 2}\label{ejercicio-2}

Se eligió la pregunta Q30: ``I think horoscopes are fun.''

\subsection{Ejercicio 3}\label{ejercicio-3}

Una forma de plantear el problema es mediante una regresión lineal donde
la variable dependiente sea la respuesta en la escala Likert, y las
covariables la edad y variables indicadoras del género. Es decir,

\[
Likert_i = \beta_{\text{intercept}} + \beta_{\text{masc}} \text{masc}_i + \beta_{\text{fem}} \text{fem}_i + \beta_{\text{edad}} \text{edad}_i + \epsilon_i
\]

El problema de este enfoque es que \(Likert_i\) toma \(1,...,5\) como
posibles valores, pero modelo lineal común asume normalidad en los
errores, y por lo tanto en la variable dependiente. En este caso, es una
variable discreta con un rango de sólo 5 valores. Por otra parte, se
asume que las distancias entre las respuestas son iguales. Por ejemplo,
podría no tener sentido considerar que la diferencia entre ``Totalmente
en desacuerdo'' y ``En desacuerdo'' sea la misma que ``En desacuerdo'' y
``Neutro''.

Otra forma es modelarlo mediante una regresión multinomial. En esta, se
modela a la probabilidad de que el individuo \(i\) responda la opción
\(j\) como

\[
\mathbb{P}(Likert_{i, j} = 1) = \text{Softmax}(z_i)_j
\]

donde \(z_i\) es un vector en \(\mathbb{R}^5\) tal que, para
\(1 \leq j \leq 5\),

\[
z_{i, j} = \beta_{\text{intercept, j}} + \beta_{\text{masc}, j} \text{masc}_i + \beta_{\text{fem}, j} \text{fem}_i + \beta_{\text{edad}, j} \text{edad}_i
\] y \(\text{Softmax} : \mathbb{R}^p \to \mathbb{R}^p\) es la función

\[
\text{Softmax}(z)_j = \frac{e^{z_j}}{\sum_{j=1}^{p} e^{z_j}}
\] Sin embargo, dicho enfoque no es el más apropiado tampoco. Esto se
debe a que la variable dependiente no es una variable categórica
nominal. Las posibles respuestas tienen un orden, que no se está
modelando.

\subsection{Ejercicio 4}\label{ejercicio-4}

El modelo de regresión ordinal es un modelo de clasificación que a
diferencia de la regresión multinomial, permite tener en cuenta un orden
de las categorías que puede tomar la variable dependiente \(Y\), como
ocurre con las escalas Likert. A diferencia de la regresión multinomial
donde se busca estimar la función de probabilidad puntual de la
distribución de categorías dados los datos mediante un modelo lineal
generalizado, ahora se busca estimar la función de distribución
acumulada de las categorías, teniendo en cuenta su orden. Es decir,
supongamos que tenemos \(K\) categorías. Dada \(1 \leq j \leq K\) una
categoría, se busca estimar \(P(Y \leq j)\). La forma en la que se
realiza esto es la siguiente:

\[
\widehat{P(Y \leq j)} := g(\theta_j - \mathbf{x}^\top \boldsymbol{\beta})
\] donde los \(\theta_j\) son parámetros denominados \emph{umbrales},
que están retringidos a \(\theta_1 < \theta_2 < \cdots < \theta_{K-1}\),
\(x\) son las covariables, y \(\beta\) son los predictores. Además,
\(g:\mathbb{R} \to [0, 1]\) es una función de link. Si la función de
link es la función logística, al modelo se lo conoce como modelo
\emph{logit}. Si la función de link es la función de distribución
acumulada de una normal estándar, se lo conoce como \emph{probit}.

\subsection{Ejercicio 5}\label{ejercicio-5}

Observamos que había valores que no estaban dentro de la escala
(particularmente el 0), por lo tanto decidimos filtrarlos ya que no los
consideramos correctos para Likert. Luego, ajustamos el modelo de
regresión ordinal \emph{logit}. No sin antes acondicionar la variable
Q30, convirtiéndola en factor ordenado, para usarla en la función polr.

\begin{Shaded}
\begin{Highlighting}[]
\NormalTok{df\_train\_30 }\OtherTok{\textless{}{-}}\NormalTok{ df\_train }\SpecialCharTok{|\textgreater{}} 
  \FunctionTok{filter}\NormalTok{(Q30 }\SpecialCharTok{!=} \DecValTok{0}\NormalTok{)}

\NormalTok{df\_test\_30 }\OtherTok{\textless{}{-}}\NormalTok{ df\_test }\SpecialCharTok{|\textgreater{}} 
  \FunctionTok{filter}\NormalTok{(Q30 }\SpecialCharTok{!=} \DecValTok{0}\NormalTok{)}

\CommentTok{\#Ordenamos en escala}
\NormalTok{df\_train\_30}\SpecialCharTok{$}\NormalTok{Q30\_factor }\OtherTok{\textless{}{-}} \FunctionTok{ordered}\NormalTok{(df\_train\_30}\SpecialCharTok{$}\NormalTok{Q30,}
                  \AttributeTok{levels =} \DecValTok{1}\SpecialCharTok{:}\DecValTok{5}\NormalTok{,}
                  \AttributeTok{labels =} \FunctionTok{c}\NormalTok{(}\StringTok{"Muy en desacuerdo"}\NormalTok{, }\StringTok{"En desacuerdo"}\NormalTok{, }\StringTok{"Neutral"}\NormalTok{, }\StringTok{"De acuerdo"}\NormalTok{, }\StringTok{"Muy de acuerdo"}\NormalTok{))}

\NormalTok{df\_test\_30}\SpecialCharTok{$}\NormalTok{Q30\_factor }\OtherTok{\textless{}{-}} \FunctionTok{ordered}\NormalTok{(df\_test\_30}\SpecialCharTok{$}\NormalTok{Q30,}
                  \AttributeTok{levels =} \DecValTok{1}\SpecialCharTok{:}\DecValTok{5}\NormalTok{,}
                  \AttributeTok{labels =} \FunctionTok{c}\NormalTok{(}\StringTok{"Muy en desacuerdo"}\NormalTok{, }\StringTok{"En desacuerdo"}\NormalTok{, }\StringTok{"Neutral"}\NormalTok{, }\StringTok{"De acuerdo"}\NormalTok{, }\StringTok{"Muy de acuerdo"}\NormalTok{))}

\NormalTok{modelo\_ordinal }\OtherTok{\textless{}{-}} \FunctionTok{polr}\NormalTok{(Q30\_factor }\SpecialCharTok{\textasciitilde{}}\NormalTok{ age, }\AttributeTok{data =}\NormalTok{ df\_train\_30, }\AttributeTok{Hess =} \ConstantTok{TRUE}\NormalTok{, }\AttributeTok{method =} \StringTok{"logistic"}\NormalTok{)}
\FunctionTok{summary}\NormalTok{(modelo\_ordinal)}
\end{Highlighting}
\end{Shaded}

\begin{verbatim}
## Call:
## polr(formula = Q30_factor ~ age, data = df_train_30, Hess = TRUE, 
##     method = "logistic")
## 
## Coefficients:
##      Value Std. Error t value
## age -0.024  0.0007619   -31.5
## 
## Intercepts:
##                                 Value    Std. Error t value 
## Muy en desacuerdo|En desacuerdo  -1.5063   0.0167   -90.1874
## En desacuerdo|Neutral            -0.9298   0.0165   -56.2942
## Neutral|De acuerdo               -0.1803   0.0164   -10.9839
## De acuerdo|Muy de acuerdo         0.8170   0.0166    49.3631
## 
## Residual Deviance: 730222.48 
## AIC: 730232.48
\end{verbatim}

Obtenemos el ajuste del modelo, que nos devuelve los umbrales y el valor
de \(\beta\) para la variable age. Vemos que: \(\beta = -0.024\) lo que
sugiere que al aumentar la edad, la probabilidad de estar en niveles más
altos en la escala Likert va disminuyendo. Es decir, es más probable que
una persona más joven esté al menos ``De Acuerdo'' que una persona más
vieja. \(t-value = -31.5\) lo que sugiere que la variable age tiene
mucha significancia en la variable respuesta ya que posee un valor
absoluto relativamente alto.

\subsection{Ejercicio 6}\label{ejercicio-6}

Ahora queremos estimar la probabilidad de que una persona de 25 años
esté al menos de acuerdo con la frase ``me gustan las armas'' (pregunta
9). Por lo tanto, como es una pregunta diferente a la que nosotros
elegimos, vamos a filtrar, desde el dataframe completo, los 0s para esta
pregunta (ya que el anterior lo habíamos filtrado por los 0s para la
pregunta Q30).

\begin{Shaded}
\begin{Highlighting}[]
\CommentTok{\# Entrenamos con todos los datos en este caso}
\NormalTok{df\_9 }\OtherTok{\textless{}{-}}\NormalTok{ df }\SpecialCharTok{|\textgreater{}} 
  \FunctionTok{filter}\NormalTok{(Q9 }\SpecialCharTok{!=} \DecValTok{0}\NormalTok{)}

\NormalTok{df\_9}\SpecialCharTok{$}\NormalTok{Q9 }\OtherTok{\textless{}{-}} \FunctionTok{ordered}\NormalTok{(df\_9}\SpecialCharTok{$}\NormalTok{Q9,}
                  \AttributeTok{levels =} \DecValTok{1}\SpecialCharTok{:}\DecValTok{5}\NormalTok{,}
                  \AttributeTok{labels =} \FunctionTok{c}\NormalTok{(}\StringTok{"Muy en desacuerdo"}\NormalTok{, }\StringTok{"En desacuerdo"}\NormalTok{, }\StringTok{"Neutral"}\NormalTok{, }\StringTok{"De acuerdo"}\NormalTok{, }\StringTok{"Muy de acuerdo"}\NormalTok{))}

\NormalTok{modelo\_9 }\OtherTok{\textless{}{-}} \FunctionTok{polr}\NormalTok{(Q9 }\SpecialCharTok{\textasciitilde{}}\NormalTok{ age, }\AttributeTok{data =}\NormalTok{ df\_9, }\AttributeTok{Hess =} \ConstantTok{TRUE}\NormalTok{, }\AttributeTok{method =} \StringTok{"logistic"}\NormalTok{)}
\FunctionTok{summary}\NormalTok{(modelo\_9)}
\end{Highlighting}
\end{Shaded}

\begin{verbatim}
## Call:
## polr(formula = Q9 ~ age, data = df_9, Hess = TRUE, method = "logistic")
## 
## Coefficients:
##         Value Std. Error t value
## age -0.003029  0.0006859  -4.416
## 
## Intercepts:
##                                 Value    Std. Error t value 
## Muy en desacuerdo|En desacuerdo  -0.7365   0.0149   -49.5847
## En desacuerdo|Neutral            -0.1889   0.0148   -12.7656
## Neutral|De acuerdo                0.5611   0.0148    37.8235
## De acuerdo|Muy de acuerdo         1.3279   0.0150    88.2849
## 
## Residual Deviance: 894778.59 
## AIC: 894788.59
\end{verbatim}

En este caso obtenemos los siguientes parámetros al ajustar:
\(\beta = -0.003\) al igual que en el caso de la pregunta Q30, al
aumentar la edad la probabilidad de estar en niveles más altos en la
escala Likert va disminuyendo. Sin embargo, en este caso el valor es más
chico, por lo que la respuesta en función de las edades se difumina un
poco ya que la distancia entre ambos es mucho más pequeña que en Q30
(que era de -0.024).

\(t-value = -4.4\) aquí también encontramos una diferencia grande con
respecto a la pregunta Q30, y es que en este caso la significancia no es
demasiado alta, lo que indica que no incorporamos demasiada información
relevante a la estimación usando la edad.

A continuación, usamos la función ``predict'', la cual estima la
probabilidad puntual de cada opción y luego sumamos las instancias ``De
acuerdo'' y ``Muy de acuerdo'' para responder a la pregunta enunciada.

\begin{Shaded}
\begin{Highlighting}[]
\NormalTok{probabilidades }\OtherTok{\textless{}{-}} \FunctionTok{predict}\NormalTok{(modelo\_9, }\AttributeTok{newdata =} \FunctionTok{data.frame}\NormalTok{(}\AttributeTok{age =} \DecValTok{25}\NormalTok{), }\AttributeTok{type =} \StringTok{"probs"}\NormalTok{)}
\FunctionTok{print}\NormalTok{(probabilidades)}
\end{Highlighting}
\end{Shaded}

\begin{verbatim}
## Muy en desacuerdo     En desacuerdo           Neutral        De acuerdo 
##         0.3405653         0.1311804         0.1822887         0.1487279 
##    Muy de acuerdo 
##         0.1972378
\end{verbatim}

\begin{Shaded}
\begin{Highlighting}[]
\NormalTok{DeAcuerdo }\OtherTok{\textless{}{-}}\NormalTok{ probabilidades[}\StringTok{"De acuerdo"}\NormalTok{]}
\NormalTok{MuyDeAcuerdo }\OtherTok{\textless{}{-}}\NormalTok{ probabilidades[}\StringTok{"Muy de acuerdo"}\NormalTok{]}
\FunctionTok{print}\NormalTok{(}\FunctionTok{paste0}\NormalTok{(}\StringTok{"Al menos de acuerdo: "}\NormalTok{, }\FunctionTok{as.numeric}\NormalTok{(DeAcuerdo }\SpecialCharTok{+}\NormalTok{ MuyDeAcuerdo)))}
\end{Highlighting}
\end{Shaded}

\begin{verbatim}
## [1] "Al menos de acuerdo: 0.34596565305203"
\end{verbatim}

\subsection{Ejercicio 7}\label{ejercicio-7}

Implementamos la función de pérdida:

\[
L(y, \hat{y}) = \frac{1}{n} \sum_{i=1}^{n} \left| y_i - \hat{y}_i \right|
\]

La misma corresponde a la función MAE (Mean Absolute Error).

\begin{Shaded}
\begin{Highlighting}[]
\NormalTok{mae }\OtherTok{\textless{}{-}} \ControlFlowTok{function}\NormalTok{(y\_true, y\_pred) \{}
  \FunctionTok{return}\NormalTok{(}\FunctionTok{mean}\NormalTok{(}\FunctionTok{abs}\NormalTok{(y\_true }\SpecialCharTok{{-}}\NormalTok{ y\_pred)))}
\NormalTok{\}}
\end{Highlighting}
\end{Shaded}

\subsection{Ejercicios 8 y 9}\label{ejercicios-8-y-9}

Entrenamos y predecimos con el modelo lineal truncado (ejercicio 8)

\begin{Shaded}
\begin{Highlighting}[]
\NormalTok{modelo\_lineal }\OtherTok{\textless{}{-}} \FunctionTok{lm}\NormalTok{(Q30 }\SpecialCharTok{\textasciitilde{}}\NormalTok{ age, }\AttributeTok{data =}\NormalTok{ df\_train\_30)}
\NormalTok{y\_pred\_train }\OtherTok{\textless{}{-}} \FunctionTok{predict}\NormalTok{(modelo\_lineal)}
\NormalTok{y\_pred\_final\_train }\OtherTok{\textless{}{-}} \FunctionTok{pmin}\NormalTok{(}\FunctionTok{pmax}\NormalTok{(}\FunctionTok{round}\NormalTok{(y\_pred\_train), }\DecValTok{1}\NormalTok{), }\DecValTok{5}\NormalTok{)}
\NormalTok{mae\_train }\OtherTok{\textless{}{-}} \FunctionTok{mean}\NormalTok{(}\FunctionTok{abs}\NormalTok{(df\_train\_30}\SpecialCharTok{$}\NormalTok{Q30 }\SpecialCharTok{{-}}\NormalTok{ y\_pred\_final\_train))}
\NormalTok{aciertos\_train }\OtherTok{\textless{}{-}} \FunctionTok{mean}\NormalTok{(df\_train\_30}\SpecialCharTok{$}\NormalTok{Q30 }\SpecialCharTok{==}\NormalTok{ y\_pred\_final\_train)}

\NormalTok{y\_pred\_test }\OtherTok{\textless{}{-}} \FunctionTok{predict}\NormalTok{(modelo\_lineal, }\AttributeTok{newdata =}\NormalTok{ df\_test\_30)}
\NormalTok{y\_pred\_final\_test }\OtherTok{\textless{}{-}} \FunctionTok{pmin}\NormalTok{(}\FunctionTok{pmax}\NormalTok{(}\FunctionTok{round}\NormalTok{(y\_pred\_test), }\DecValTok{1}\NormalTok{), }\DecValTok{5}\NormalTok{)}
\NormalTok{mae\_test }\OtherTok{\textless{}{-}} \FunctionTok{mean}\NormalTok{(}\FunctionTok{abs}\NormalTok{(df\_test\_30}\SpecialCharTok{$}\NormalTok{Q30 }\SpecialCharTok{{-}}\NormalTok{ y\_pred\_final\_test))}
\NormalTok{aciertos\_test }\OtherTok{\textless{}{-}} \FunctionTok{mean}\NormalTok{(df\_test\_30}\SpecialCharTok{$}\NormalTok{Q30 }\SpecialCharTok{==}\NormalTok{ y\_pred\_final\_test)}
\FunctionTok{print}\NormalTok{(}\StringTok{"Resultados para modelo lineal truncado:"}\NormalTok{)}
\end{Highlighting}
\end{Shaded}

\begin{verbatim}
## [1] "Resultados para modelo lineal truncado:"
\end{verbatim}

\begin{Shaded}
\begin{Highlighting}[]
\FunctionTok{list}\NormalTok{(}\AttributeTok{MAE\_train =}\NormalTok{ mae\_train, }\AttributeTok{MAE\_test =}\NormalTok{ mae\_test, }\AttributeTok{exactitud\_train =}\NormalTok{ aciertos\_train, }\AttributeTok{exactitud\_test =}\NormalTok{ aciertos\_test)}
\end{Highlighting}
\end{Shaded}

\begin{verbatim}
## $MAE_train
## [1] 1.295728
## 
## $MAE_test
## [1] 1.289727
## 
## $exactitud_train
## [1] 0.1845297
## 
## $exactitud_test
## [1] 0.1864128
\end{verbatim}

Obtenemos así, los siguientes resultados:

\[
\begin{table}[H]
\centering
\begin{tabular}{|l|c|c|}
\hline
\textbf{Métrica} & \textbf{Train} & \textbf{Test} \\
\hline
MAE & 1.3 & 1.29 \\
\hline
Exactitud & 0.18 & 0.19 \\
\hline
\end{tabular}
\caption{Regresion Lineal Truncada: Resultados de MAE y Exactitud en los conjuntos de entrenamiento y prueba}
\end{table}
\]

Entrenamos y predecimos con el modelo de regresión ordinal (ejercicio 9)

\begin{Shaded}
\begin{Highlighting}[]
\NormalTok{y\_pred\_train }\OtherTok{\textless{}{-}} \FunctionTok{as.numeric}\NormalTok{(}\FunctionTok{predict}\NormalTok{(modelo\_ordinal))}
\NormalTok{mae\_train }\OtherTok{\textless{}{-}} \FunctionTok{mean}\NormalTok{(}\FunctionTok{abs}\NormalTok{(df\_train\_30}\SpecialCharTok{$}\NormalTok{Q30 }\SpecialCharTok{{-}}\NormalTok{ y\_pred\_train))}
\NormalTok{aciertos\_train }\OtherTok{\textless{}{-}} \FunctionTok{mean}\NormalTok{(df\_train\_30}\SpecialCharTok{$}\NormalTok{Q30 }\SpecialCharTok{==}\NormalTok{ y\_pred\_train)}

\NormalTok{y\_pred\_test }\OtherTok{\textless{}{-}} \FunctionTok{as.numeric}\NormalTok{(}\FunctionTok{predict}\NormalTok{(modelo\_ordinal, }\AttributeTok{newdata =}\NormalTok{ df\_test\_30))}
\CommentTok{\# y\_pred\_final\_test \textless{}{-} pmin(pmax(round(y\_pred\_test), 1), 5)   innecesario, y\_pred\_test ya es un \textquotesingle{}factor\textquotesingle{}, no es como el modelo lineal}
\NormalTok{mae\_test }\OtherTok{\textless{}{-}} \FunctionTok{mean}\NormalTok{(}\FunctionTok{abs}\NormalTok{(df\_test\_30}\SpecialCharTok{$}\NormalTok{Q30 }\SpecialCharTok{{-}}\NormalTok{ y\_pred\_test))}
\NormalTok{aciertos\_test }\OtherTok{\textless{}{-}} \FunctionTok{mean}\NormalTok{(df\_test\_30}\SpecialCharTok{$}\NormalTok{Q30 }\SpecialCharTok{==}\NormalTok{ y\_pred\_test)}
\FunctionTok{print}\NormalTok{(}\StringTok{"Resultados para modelo ordinal logístico:"}\NormalTok{)}
\end{Highlighting}
\end{Shaded}

\begin{verbatim}
## [1] "Resultados para modelo ordinal logístico:"
\end{verbatim}

\begin{Shaded}
\begin{Highlighting}[]
\FunctionTok{list}\NormalTok{(}\AttributeTok{MAE\_train =}\NormalTok{ mae\_train, }\AttributeTok{MAE\_test =}\NormalTok{ mae\_test, }\AttributeTok{exactitud\_train =}\NormalTok{ aciertos\_train, }\AttributeTok{exactitud\_test =}\NormalTok{ aciertos\_test)}
\end{Highlighting}
\end{Shaded}

\begin{verbatim}
## $MAE_train
## [1] 1.945435
## 
## $MAE_test
## [1] 1.956452
## 
## $exactitud_train
## [1] 0.2725172
## 
## $exactitud_test
## [1] 0.2680398
\end{verbatim}

\$\$

\begin{table}[H]
\centering
\begin{tabular}{|l|c|c|}
\hline
\textbf{Métrica} & \textbf{Train} & \textbf{Test} \\
\hline
MAE & 1.95 & 1.96 \\
\hline
Exactitud & 0.27 & 0.27 \\
\hline
\end{tabular}
\caption{Regresion Ordinal: Resultados de MAE y Exactitud en los conjuntos de entrenamiento y prueba}
\end{table}

\$\$

Los modelos tuvieron un mal desempeño, dado que tener una distancia
mayor a 1 en la escala Likert puede ser un cambio de respuesta de una
categoría a otra y eso puede significar pasar, por ejemplo, de un
``Neutro'' a un ``De acuerdo'' (un abismo de diferencia). Veamos también
que el modelo lineal truncado es incluso mejor que el modelo ordinal, lo
que es a priori extraño dado que el modelo lineal no está acotado ni
tiene una estructura demasiado compatible con la escala Likert como sí
lo tiene el modelo ordinal.

Habiendo visto estos resultados, lo más probable es que esté pasando una
de estas dos cosas (o ambas): o hay poca varianza en las respuestas a
`Q30' o la edad no es un buen predictor de `Q30'.

Con esto en mente, decidimos ver qué porcentaje de personas respondió
cada valor de la escala para `Q30' y también visualizar la distribución
de las edades de las personas que al menos estaban de acuerdo vs las que
no.

\begin{Shaded}
\begin{Highlighting}[]
\CommentTok{\# Tabulación y porcentaje}
\NormalTok{tabla }\OtherTok{\textless{}{-}} \FunctionTok{table}\NormalTok{((df }\SpecialCharTok{|\textgreater{}} \FunctionTok{filter}\NormalTok{(Q30 }\SpecialCharTok{!=} \DecValTok{0}\NormalTok{))}\SpecialCharTok{$}\NormalTok{Q30)}
\NormalTok{porcentaje }\OtherTok{\textless{}{-}} \FunctionTok{prop.table}\NormalTok{(tabla) }\SpecialCharTok{*} \DecValTok{100}
\NormalTok{porcentaje\_formateado }\OtherTok{\textless{}{-}} \FunctionTok{round}\NormalTok{(porcentaje, }\DecValTok{2}\NormalTok{)}
\FunctionTok{data.frame}\NormalTok{(}\AttributeTok{Respuesta =} \FunctionTok{names}\NormalTok{(tabla), }\AttributeTok{Porcentaje =}\NormalTok{ porcentaje\_formateado)}
\end{Highlighting}
\end{Shaded}

\begin{verbatim}
##   Respuesta Porcentaje.Var1 Porcentaje.Freq
## 1         1               1           26.78
## 2         2               2           12.62
## 3         3               3           18.49
## 4         4               4           20.94
## 5         5               5           21.16
\end{verbatim}

\begin{Shaded}
\begin{Highlighting}[]
\NormalTok{df}\SpecialCharTok{$}\NormalTok{Q30\_grupo }\OtherTok{\textless{}{-}} \FunctionTok{ifelse}\NormalTok{(df}\SpecialCharTok{$}\NormalTok{Q30 }\SpecialCharTok{\%in\%} \DecValTok{4}\SpecialCharTok{:}\DecValTok{5}\NormalTok{, }\StringTok{"De acuerdo (4 o 5)"}\NormalTok{, }\StringTok{"No de acuerdo (1 a 3)"}\NormalTok{)}
\NormalTok{df\_box }\OtherTok{\textless{}{-}}\NormalTok{ df[}\SpecialCharTok{!}\FunctionTok{is.na}\NormalTok{(df}\SpecialCharTok{$}\NormalTok{age) }\SpecialCharTok{\&} \SpecialCharTok{!}\FunctionTok{is.na}\NormalTok{(df}\SpecialCharTok{$}\NormalTok{Q30\_grupo), ]}

\CommentTok{\# Boxplot}
\FunctionTok{boxplot}\NormalTok{(age }\SpecialCharTok{\textasciitilde{}}\NormalTok{ Q30\_grupo, }\AttributeTok{data =}\NormalTok{ df\_box,}
        \AttributeTok{col =} \FunctionTok{c}\NormalTok{(}\StringTok{"\#FFA07A"}\NormalTok{, }\StringTok{"\#90EE90"}\NormalTok{),}
        \AttributeTok{main =} \StringTok{"Distribución de edad por respuesta a Q30"}\NormalTok{,}
        \AttributeTok{ylab =} \StringTok{"Edad"}\NormalTok{,}
        \AttributeTok{xlab =} \StringTok{"Grupo de respuesta"}\NormalTok{)}
\end{Highlighting}
\end{Shaded}

\includegraphics{tp1_files/figure-latex/unnamed-chunk-9-1.pdf}

Acá se ve claramente cómo casi no hay una diferencia clara por edad. Por
lo tanto, es razonable que ninguno de los dos modelos (el lineal
truncado y el ordinal) dé buenos resultados. Por otra parte, veamos la
distribución de las clases en el conjunto de test

\begin{Shaded}
\begin{Highlighting}[]
\FunctionTok{hist}\NormalTok{(df\_test\_30}\SpecialCharTok{$}\NormalTok{Q30, }\AttributeTok{main=}\StringTok{"Distribución de clases de Q30"}\NormalTok{, }\AttributeTok{xlab =} \StringTok{"Respuesta"}\NormalTok{)}
\end{Highlighting}
\end{Shaded}

\includegraphics{tp1_files/figure-latex/unnamed-chunk-10-1.pdf}

A pesar de que hay un desbalance de clases especialmente entre las
respuestas \textbf{1} y \textbf{2}, no parecería que el problema viene
por este lado. Por lo tanto, para verificar que el problema en la
covariable, decidimos crear datos sintéticos donde se observen
diferencias en las distribuciones de edades, y ver si el modelo ajusta
de manera razonable a los datos. Para descartar que sea un problema de
un posible problema del desbalance de clases, se decidió crear la misma
distribución de clases que el conjunto de test.

\begin{Shaded}
\begin{Highlighting}[]
\FunctionTok{set.seed}\NormalTok{(}\DecValTok{123}\NormalTok{)}

\NormalTok{n\_total }\OtherTok{\textless{}{-}} \DecValTok{1000}

\CommentTok{\#n\_per\_group \textless{}{-} n\_total / 5}
\NormalTok{n\_per\_group }\OtherTok{\textless{}{-}}\NormalTok{ n\_total}\SpecialCharTok{*}\FunctionTok{table}\NormalTok{(df\_test\_30}\SpecialCharTok{$}\NormalTok{Q30)}\SpecialCharTok{/}\FunctionTok{nrow}\NormalTok{(df\_test\_30)}

\NormalTok{likert\_levels }\OtherTok{\textless{}{-}} \DecValTok{1}\SpecialCharTok{:}\DecValTok{5}
\NormalTok{age\_means }\OtherTok{\textless{}{-}} \FunctionTok{c}\NormalTok{(}\DecValTok{95}\NormalTok{, }\DecValTok{70}\NormalTok{, }\DecValTok{50}\NormalTok{, }\DecValTok{30}\NormalTok{, }\DecValTok{15}\NormalTok{)}
\NormalTok{age\_sds   }\OtherTok{\textless{}{-}} \FunctionTok{c}\NormalTok{(}\DecValTok{5}\NormalTok{, }\DecValTok{10}\NormalTok{, }\DecValTok{8}\NormalTok{, }\DecValTok{6}\NormalTok{, }\DecValTok{3}\NormalTok{)}

\NormalTok{data\_list }\OtherTok{\textless{}{-}} \FunctionTok{lapply}\NormalTok{(}\DecValTok{1}\SpecialCharTok{:}\DecValTok{5}\NormalTok{, }\ControlFlowTok{function}\NormalTok{(i) \{}
\NormalTok{  ages }\OtherTok{\textless{}{-}} \FunctionTok{rnorm}\NormalTok{(n\_per\_group[i], }\AttributeTok{mean =}\NormalTok{ age\_means[i], }\AttributeTok{sd =}\NormalTok{ age\_sds[i])}
\NormalTok{  ages }\OtherTok{\textless{}{-}} \FunctionTok{pmin}\NormalTok{(}\FunctionTok{pmax}\NormalTok{(ages, }\DecValTok{10}\NormalTok{), }\DecValTok{110}\NormalTok{)}
  \FunctionTok{data.frame}\NormalTok{(}
    \AttributeTok{Q\_synthetic =} \FunctionTok{rep}\NormalTok{(likert\_levels[i], n\_per\_group[i]),}
    \AttributeTok{age =}\NormalTok{ ages}
\NormalTok{  )}
\NormalTok{\})}

\NormalTok{df\_synthetic }\OtherTok{\textless{}{-}} \FunctionTok{do.call}\NormalTok{(rbind, data\_list)}

\FunctionTok{boxplot}\NormalTok{(age }\SpecialCharTok{\textasciitilde{}}\NormalTok{ Q\_synthetic, }\AttributeTok{data =}\NormalTok{ df\_synthetic, }\AttributeTok{main =} \StringTok{"Edad por respuesta Likert"}\NormalTok{, }\AttributeTok{xlab =} \StringTok{"Q\_synthetic"}\NormalTok{, }\AttributeTok{ylab =} \StringTok{"Edad"}\NormalTok{)}
\end{Highlighting}
\end{Shaded}

\includegraphics{tp1_files/figure-latex/unnamed-chunk-11-1.pdf}

\begin{Shaded}
\begin{Highlighting}[]
\NormalTok{split }\OtherTok{\textless{}{-}} \FunctionTok{initial\_split}\NormalTok{(df\_synthetic, }\AttributeTok{prop =} \FloatTok{0.8}\NormalTok{, }\AttributeTok{strata =}\NormalTok{ Q\_synthetic)}
\NormalTok{train\_data\_synth }\OtherTok{\textless{}{-}} \FunctionTok{training}\NormalTok{(split)}
\NormalTok{test\_data\_synth  }\OtherTok{\textless{}{-}} \FunctionTok{testing}\NormalTok{(split)}

\NormalTok{train\_data\_synth}\SpecialCharTok{$}\NormalTok{Q\_synthetic }\OtherTok{\textless{}{-}} \FunctionTok{factor}\NormalTok{(train\_data\_synth}\SpecialCharTok{$}\NormalTok{Q\_synthetic, }\AttributeTok{ordered =} \ConstantTok{TRUE}\NormalTok{)}
\NormalTok{test\_data\_synth}\SpecialCharTok{$}\NormalTok{Q\_synthetic  }\OtherTok{\textless{}{-}} \FunctionTok{factor}\NormalTok{(test\_data\_synth}\SpecialCharTok{$}\NormalTok{Q\_synthetic, }\AttributeTok{ordered =} \ConstantTok{TRUE}\NormalTok{)}

\NormalTok{modelo\_ordinal\_synth }\OtherTok{\textless{}{-}} \FunctionTok{polr}\NormalTok{(Q\_synthetic }\SpecialCharTok{\textasciitilde{}}\NormalTok{ age, }\AttributeTok{data =}\NormalTok{ train\_data\_synth, }\AttributeTok{Hess =} \ConstantTok{TRUE}\NormalTok{)}

\NormalTok{preds\_synth }\OtherTok{\textless{}{-}} \FunctionTok{predict}\NormalTok{(modelo\_ordinal\_synth, }\AttributeTok{newdata =}\NormalTok{ test\_data\_synth)}
\NormalTok{mae\_synth }\OtherTok{\textless{}{-}} \FunctionTok{mae}\NormalTok{(}\FunctionTok{as.numeric}\NormalTok{(preds\_synth), }\FunctionTok{as.numeric}\NormalTok{(test\_data\_synth}\SpecialCharTok{$}\NormalTok{Q\_synthetic))}
\end{Highlighting}
\end{Shaded}

El MAE obtenido en el conjunto de test es de

\begin{Shaded}
\begin{Highlighting}[]
\FunctionTok{cat}\NormalTok{(}\StringTok{"MAE:"}\NormalTok{, mae\_synth, }\StringTok{"}\SpecialCharTok{\textbackslash{}n}\StringTok{"}\NormalTok{)}
\end{Highlighting}
\end{Shaded}

\begin{verbatim}
## MAE: 0.1039604
\end{verbatim}

Esto implica un buen ajuste para los datos sintéticos, lo que supone que
el principal problema viene por la capacidad predictiva de \(\beta\).

\subsubsection{Maybe delete}\label{maybe-delete}

Ahora, podemos buscar alguna pregunta que sí muestre una diferencia más
contundente por edad. Para eso, podemos correr este programa (en vez de
mirar un boxplots por cada pregunta) que nos ordene a las preguntas en
base a la diferencia de edad que tiene la población de acuerdo y la que
no. Para comparar las edades de las poblaciones, vamos a intentar
únicamente comparando medias (puede que no haga falta más que esto).

\begin{Shaded}
\begin{Highlighting}[]
\NormalTok{max\_dif }\OtherTok{\textless{}{-}} \DecValTok{0}
\NormalTok{max\_dif\_col }\OtherTok{\textless{}{-}} \StringTok{""}

\ControlFlowTok{for}\NormalTok{ (col }\ControlFlowTok{in} \FunctionTok{colnames}\NormalTok{(df)[}\DecValTok{1}\SpecialCharTok{:}\DecValTok{44}\NormalTok{]) \{}
\NormalTok{  df}\SpecialCharTok{$}\NormalTok{grupos }\OtherTok{\textless{}{-}} \FunctionTok{ifelse}\NormalTok{(df[[col]] }\SpecialCharTok{\%in\%} \DecValTok{4}\SpecialCharTok{:}\DecValTok{5}\NormalTok{, }\StringTok{"De acuerdo"}\NormalTok{, }\StringTok{"No de acuerdo"}\NormalTok{)}
\NormalTok{  df\_box }\OtherTok{\textless{}{-}}\NormalTok{ df[}\SpecialCharTok{!}\FunctionTok{is.na}\NormalTok{(df}\SpecialCharTok{$}\NormalTok{age) }\SpecialCharTok{\&} \SpecialCharTok{!}\FunctionTok{is.na}\NormalTok{(df[[col]]), ]}
\NormalTok{  media\_dif }\OtherTok{\textless{}{-}} \FunctionTok{abs}\NormalTok{(}\FunctionTok{mean}\NormalTok{(df\_box}\SpecialCharTok{$}\NormalTok{age[df\_box}\SpecialCharTok{$}\NormalTok{grupos }\SpecialCharTok{==} \StringTok{"De acuerdo"}\NormalTok{]) }\SpecialCharTok{{-}} \FunctionTok{mean}\NormalTok{(df\_box}\SpecialCharTok{$}\NormalTok{age[df\_box}\SpecialCharTok{$}\NormalTok{grupos }\SpecialCharTok{==} \StringTok{"No de acuerdo"}\NormalTok{]))}
  \ControlFlowTok{if}\NormalTok{ (media\_dif }\SpecialCharTok{\textgreater{}}\NormalTok{ max\_dif) \{}
\NormalTok{    max\_dif\_col }\OtherTok{\textless{}{-}}\NormalTok{ col}
\NormalTok{    max\_dif }\OtherTok{\textless{}{-}}\NormalTok{ media\_dif}
\NormalTok{  \}}
\NormalTok{\}}

\FunctionTok{print}\NormalTok{(max\_dif\_col)}
\end{Highlighting}
\end{Shaded}

\begin{verbatim}
## [1] "Q23"
\end{verbatim}

\begin{Shaded}
\begin{Highlighting}[]
\FunctionTok{print}\NormalTok{(max\_dif)}
\end{Highlighting}
\end{Shaded}

\begin{verbatim}
## [1] 1.669733
\end{verbatim}

El programa indica que la pregunta Q23 (``I playfully insult my
friends'') maximiza la diferencia de medias de las poblaciones a favor y
en contra. Veámoslo en el mismo gráfico que hicimos para Q30.

\begin{Shaded}
\begin{Highlighting}[]
\NormalTok{df}\SpecialCharTok{$}\NormalTok{Q23\_grupo }\OtherTok{\textless{}{-}} \FunctionTok{ifelse}\NormalTok{(df}\SpecialCharTok{$}\NormalTok{Q23 }\SpecialCharTok{\%in\%} \DecValTok{4}\SpecialCharTok{:}\DecValTok{5}\NormalTok{, }\StringTok{"De acuerdo (4 o 5)"}\NormalTok{, }\StringTok{"No de acuerdo (1 a 3)"}\NormalTok{)}
\NormalTok{df\_box }\OtherTok{\textless{}{-}}\NormalTok{ df[}\SpecialCharTok{!}\FunctionTok{is.na}\NormalTok{(df}\SpecialCharTok{$}\NormalTok{age) }\SpecialCharTok{\&} \SpecialCharTok{!}\FunctionTok{is.na}\NormalTok{(df}\SpecialCharTok{$}\NormalTok{Q23\_grupo), ]}

\FunctionTok{boxplot}\NormalTok{(age }\SpecialCharTok{\textasciitilde{}}\NormalTok{ Q23\_grupo, }\AttributeTok{data =}\NormalTok{ df\_box,}
        \AttributeTok{col =} \FunctionTok{c}\NormalTok{(}\StringTok{"\#FFA07A"}\NormalTok{, }\StringTok{"\#90EE90"}\NormalTok{),}
        \AttributeTok{main =} \StringTok{"Distribución de edad por respuesta a Q23"}\NormalTok{,}
        \AttributeTok{ylab =} \StringTok{"Edad"}\NormalTok{,}
        \AttributeTok{xlab =} \StringTok{"Grupo de respuesta"}\NormalTok{)}
\end{Highlighting}
\end{Shaded}

\includegraphics{tp1_files/figure-latex/unnamed-chunk-14-1.pdf}

\begin{Shaded}
\begin{Highlighting}[]
\FunctionTok{table}\NormalTok{(df}\SpecialCharTok{$}\NormalTok{Q23\_grupo)}
\end{Highlighting}
\end{Shaded}

\begin{verbatim}
## 
##    De acuerdo (4 o 5) No de acuerdo (1 a 3) 
##                228581                 60882
\end{verbatim}

Podemos observar que ahora sí hay una diferencia entre las edades de las
poblaciones de acuerdo y en desacuerdo. Veamos si los modelos predicen
mejor la adhesión o no a esta pregunta `Q23'.

\begin{Shaded}
\begin{Highlighting}[]
\NormalTok{df\_train\_23 }\OtherTok{\textless{}{-}}\NormalTok{ df\_train }\SpecialCharTok{|\textgreater{}} 
  \FunctionTok{filter}\NormalTok{(Q23 }\SpecialCharTok{!=} \DecValTok{0}\NormalTok{)}

\NormalTok{df\_test\_23 }\OtherTok{\textless{}{-}}\NormalTok{ df\_test }\SpecialCharTok{|\textgreater{}} 
  \FunctionTok{filter}\NormalTok{(Q23 }\SpecialCharTok{!=} \DecValTok{0}\NormalTok{)}

\CommentTok{\# Ordenamos en escala}
\NormalTok{df\_train\_23}\SpecialCharTok{$}\NormalTok{Q23\_factor }\OtherTok{\textless{}{-}} \FunctionTok{ordered}\NormalTok{(df\_train\_23}\SpecialCharTok{$}\NormalTok{Q23,}
                  \AttributeTok{levels =} \DecValTok{1}\SpecialCharTok{:}\DecValTok{5}\NormalTok{,}
                  \AttributeTok{labels =} \FunctionTok{c}\NormalTok{(}\StringTok{"Muy en desacuerdo"}\NormalTok{, }\StringTok{"En desacuerdo"}\NormalTok{, }\StringTok{"Neutral"}\NormalTok{, }\StringTok{"De acuerdo"}\NormalTok{, }\StringTok{"Muy de acuerdo"}\NormalTok{))}

\NormalTok{df\_test\_23}\SpecialCharTok{$}\NormalTok{Q23\_factor }\OtherTok{\textless{}{-}} \FunctionTok{ordered}\NormalTok{(df\_test\_23}\SpecialCharTok{$}\NormalTok{Q23,}
                  \AttributeTok{levels =} \DecValTok{1}\SpecialCharTok{:}\DecValTok{5}\NormalTok{,}
                  \AttributeTok{labels =} \FunctionTok{c}\NormalTok{(}\StringTok{"Muy en desacuerdo"}\NormalTok{, }\StringTok{"En desacuerdo"}\NormalTok{, }\StringTok{"Neutral"}\NormalTok{, }\StringTok{"De acuerdo"}\NormalTok{, }\StringTok{"Muy de acuerdo"}\NormalTok{))}

\NormalTok{modelo\_ordinal\_23 }\OtherTok{\textless{}{-}} \FunctionTok{polr}\NormalTok{(Q23\_factor }\SpecialCharTok{\textasciitilde{}}\NormalTok{ age, }\AttributeTok{data =}\NormalTok{ df\_train\_23, }\AttributeTok{Hess =} \ConstantTok{TRUE}\NormalTok{, }\AttributeTok{method =} \StringTok{"logistic"}\NormalTok{)}
\FunctionTok{summary}\NormalTok{(modelo\_ordinal\_23)}
\end{Highlighting}
\end{Shaded}

\begin{verbatim}
## Call:
## polr(formula = Q23_factor ~ age, data = df_train_23, Hess = TRUE, 
##     method = "logistic")
## 
## Coefficients:
##       Value Std. Error t value
## age -0.0682  0.0008121  -83.98
## 
## Intercepts:
##                                 Value     Std. Error t value  
## Muy en desacuerdo|En desacuerdo   -4.4000    0.0202  -217.4886
## En desacuerdo|Neutral             -3.5151    0.0189  -186.1118
## Neutral|De acuerdo                -2.7912    0.0183  -152.9111
## De acuerdo|Muy de acuerdo         -1.5235    0.0175   -86.8804
## 
## Residual Deviance: 565017.63 
## AIC: 565027.63
\end{verbatim}

\begin{Shaded}
\begin{Highlighting}[]
\NormalTok{modelo\_lineal }\OtherTok{\textless{}{-}} \FunctionTok{lm}\NormalTok{(Q23 }\SpecialCharTok{\textasciitilde{}}\NormalTok{ age, }\AttributeTok{data =}\NormalTok{ df\_train\_23)}
\NormalTok{y\_pred\_train }\OtherTok{\textless{}{-}} \FunctionTok{predict}\NormalTok{(modelo\_lineal)}
\NormalTok{y\_pred\_final\_train }\OtherTok{\textless{}{-}} \FunctionTok{pmin}\NormalTok{(}\FunctionTok{pmax}\NormalTok{(}\FunctionTok{round}\NormalTok{(y\_pred\_train), }\DecValTok{1}\NormalTok{), }\DecValTok{5}\NormalTok{)}
\NormalTok{mae\_train }\OtherTok{\textless{}{-}} \FunctionTok{mean}\NormalTok{(}\FunctionTok{abs}\NormalTok{(df\_train\_23}\SpecialCharTok{$}\NormalTok{Q23 }\SpecialCharTok{{-}}\NormalTok{ y\_pred\_final\_train))}
\NormalTok{aciertos\_train }\OtherTok{\textless{}{-}} \FunctionTok{mean}\NormalTok{(df\_train\_23}\SpecialCharTok{$}\NormalTok{Q23 }\SpecialCharTok{==}\NormalTok{ y\_pred\_final\_train)}

\NormalTok{y\_pred\_test }\OtherTok{\textless{}{-}} \FunctionTok{predict}\NormalTok{(modelo\_lineal, }\AttributeTok{newdata =}\NormalTok{ df\_test\_23)}
\NormalTok{y\_pred\_final\_test }\OtherTok{\textless{}{-}} \FunctionTok{pmin}\NormalTok{(}\FunctionTok{pmax}\NormalTok{(}\FunctionTok{round}\NormalTok{(y\_pred\_test), }\DecValTok{1}\NormalTok{), }\DecValTok{5}\NormalTok{)}
\NormalTok{mae\_test }\OtherTok{\textless{}{-}} \FunctionTok{mean}\NormalTok{(}\FunctionTok{abs}\NormalTok{(df\_test\_23}\SpecialCharTok{$}\NormalTok{Q23 }\SpecialCharTok{{-}}\NormalTok{ y\_pred\_final\_test))}
\NormalTok{aciertos\_test }\OtherTok{\textless{}{-}} \FunctionTok{mean}\NormalTok{(df\_test\_23}\SpecialCharTok{$}\NormalTok{Q23 }\SpecialCharTok{==}\NormalTok{ y\_pred\_final\_test)}
\FunctionTok{print}\NormalTok{(}\StringTok{"Resultados para modelo lineal truncado:"}\NormalTok{)}
\end{Highlighting}
\end{Shaded}

\begin{verbatim}
## [1] "Resultados para modelo lineal truncado:"
\end{verbatim}

\begin{Shaded}
\begin{Highlighting}[]
\FunctionTok{list}\NormalTok{(}\AttributeTok{MAE\_train =}\NormalTok{ mae\_train, }\AttributeTok{MAE\_test =}\NormalTok{ mae\_test, }\AttributeTok{exactitud\_train =}\NormalTok{ aciertos\_train, }\AttributeTok{exactitud\_test =}\NormalTok{ aciertos\_test)}
\end{Highlighting}
\end{Shaded}

\begin{verbatim}
## $MAE_train
## [1] 0.8977559
## 
## $MAE_test
## [1] 0.9003204
## 
## $exactitud_train
## [1] 0.2671684
## 
## $exactitud_test
## [1] 0.2653909
\end{verbatim}

\begin{Shaded}
\begin{Highlighting}[]
\NormalTok{y\_pred\_train }\OtherTok{\textless{}{-}} \FunctionTok{as.numeric}\NormalTok{(}\FunctionTok{predict}\NormalTok{(modelo\_ordinal\_23))}
\NormalTok{mae\_train }\OtherTok{\textless{}{-}} \FunctionTok{mean}\NormalTok{(}\FunctionTok{abs}\NormalTok{(df\_train\_23}\SpecialCharTok{$}\NormalTok{Q23 }\SpecialCharTok{{-}}\NormalTok{ y\_pred\_train))}
\NormalTok{aciertos\_train }\OtherTok{\textless{}{-}} \FunctionTok{mean}\NormalTok{(df\_train\_23}\SpecialCharTok{$}\NormalTok{Q23 }\SpecialCharTok{==}\NormalTok{ y\_pred\_train)}

\NormalTok{y\_pred\_test }\OtherTok{\textless{}{-}} \FunctionTok{as.numeric}\NormalTok{(}\FunctionTok{predict}\NormalTok{(modelo\_ordinal\_23, }\AttributeTok{newdata =}\NormalTok{ df\_test\_23))}
\CommentTok{\# y\_pred\_final\_test \textless{}{-} pmin(pmax(round(y\_pred\_test), 1), 5)   innecesario, y\_pred\_test ya es un \textquotesingle{}factor\textquotesingle{}, no es como el modelo lineal}
\NormalTok{mae\_test }\OtherTok{\textless{}{-}} \FunctionTok{mean}\NormalTok{(}\FunctionTok{abs}\NormalTok{(df\_test\_23}\SpecialCharTok{$}\NormalTok{Q23 }\SpecialCharTok{{-}}\NormalTok{ y\_pred\_test))}
\NormalTok{aciertos\_test }\OtherTok{\textless{}{-}} \FunctionTok{mean}\NormalTok{(df\_test\_23}\SpecialCharTok{$}\NormalTok{Q23 }\SpecialCharTok{==}\NormalTok{ y\_pred\_test)}
\FunctionTok{print}\NormalTok{(}\StringTok{"Resultados para modelo ordinal logístico:"}\NormalTok{)}
\end{Highlighting}
\end{Shaded}

\begin{verbatim}
## [1] "Resultados para modelo ordinal logístico:"
\end{verbatim}

\begin{Shaded}
\begin{Highlighting}[]
\FunctionTok{list}\NormalTok{(}\AttributeTok{MAE\_train =}\NormalTok{ mae\_train, }\AttributeTok{MAE\_test =}\NormalTok{ mae\_test, }\AttributeTok{exactitud\_train =}\NormalTok{ aciertos\_train, }\AttributeTok{exactitud\_test =}\NormalTok{ aciertos\_test)}
\end{Highlighting}
\end{Shaded}

\begin{verbatim}
## $MAE_train
## [1] 0.8439663
## 
## $MAE_test
## [1] 0.8471729
## 
## $exactitud_train
## [1] 0.524788
## 
## $exactitud_test
## [1] 0.5247034
\end{verbatim}

¿Y si en vez de medir la exactitud en base a qué porcentaje de la
población el modelo predijo exactamente su respuesta (1 a 5),
intentaramos medir a cuántos le acierta sobre si están al menos a favor
o no?

\begin{Shaded}
\begin{Highlighting}[]
\CommentTok{\# Verdadero sentimiento}
\NormalTok{grupo\_real\_train }\OtherTok{\textless{}{-}} \FunctionTok{ifelse}\NormalTok{(df\_train\_23}\SpecialCharTok{$}\NormalTok{Q23 }\SpecialCharTok{\%in\%} \DecValTok{4}\SpecialCharTok{:}\DecValTok{5}\NormalTok{, }\StringTok{"a\_favor"}\NormalTok{, }\StringTok{"no\_a\_favor"}\NormalTok{)}
\NormalTok{grupo\_real\_test  }\OtherTok{\textless{}{-}} \FunctionTok{ifelse}\NormalTok{(df\_test\_23}\SpecialCharTok{$}\NormalTok{Q23  }\SpecialCharTok{\%in\%} \DecValTok{4}\SpecialCharTok{:}\DecValTok{5}\NormalTok{, }\StringTok{"a\_favor"}\NormalTok{, }\StringTok{"no\_a\_favor"}\NormalTok{)}

\CommentTok{\# Predicción de sentimiento}
\NormalTok{grupo\_pred\_train }\OtherTok{\textless{}{-}} \FunctionTok{ifelse}\NormalTok{(y\_pred\_train }\SpecialCharTok{\%in\%} \DecValTok{4}\SpecialCharTok{:}\DecValTok{5}\NormalTok{, }\StringTok{"a\_favor"}\NormalTok{, }\StringTok{"no\_a\_favor"}\NormalTok{)}
\NormalTok{grupo\_pred\_test  }\OtherTok{\textless{}{-}} \FunctionTok{ifelse}\NormalTok{(y\_pred\_test  }\SpecialCharTok{\%in\%} \DecValTok{4}\SpecialCharTok{:}\DecValTok{5}\NormalTok{, }\StringTok{"a\_favor"}\NormalTok{, }\StringTok{"no\_a\_favor"}\NormalTok{)}

\CommentTok{\# 3. Exactitud en clasificar a favor vs no a favor}
\NormalTok{acierto\_a\_favor\_train }\OtherTok{\textless{}{-}} \FunctionTok{mean}\NormalTok{(grupo\_real\_train }\SpecialCharTok{==}\NormalTok{ grupo\_pred\_train)}
\NormalTok{acierto\_a\_favor\_test  }\OtherTok{\textless{}{-}} \FunctionTok{mean}\NormalTok{(grupo\_real\_test  }\SpecialCharTok{==}\NormalTok{ grupo\_pred\_test)}

\CommentTok{\# 4. Mostrar todo}
\FunctionTok{print}\NormalTok{(}\StringTok{"Resultados para modelo ordinal logístico:"}\NormalTok{)}
\end{Highlighting}
\end{Shaded}

\begin{verbatim}
## [1] "Resultados para modelo ordinal logístico:"
\end{verbatim}

\begin{Shaded}
\begin{Highlighting}[]
\FunctionTok{list}\NormalTok{(}
  \AttributeTok{MAE\_train =}\NormalTok{ mae\_train,}
  \AttributeTok{MAE\_test =}\NormalTok{ mae\_test,}
  \AttributeTok{exactitud\_train =}\NormalTok{ aciertos\_train,}
  \AttributeTok{exactitud\_test =}\NormalTok{ aciertos\_test,}
  \AttributeTok{exactitud\_grupo\_train =}\NormalTok{ acierto\_a\_favor\_train,}
  \AttributeTok{exactitud\_grupo\_test =}\NormalTok{ acierto\_a\_favor\_test}
\NormalTok{)}
\end{Highlighting}
\end{Shaded}

\begin{verbatim}
## $MAE_train
## [1] 0.8439663
## 
## $MAE_test
## [1] 0.8471729
## 
## $exactitud_train
## [1] 0.524788
## 
## $exactitud_test
## [1] 0.5247034
## 
## $exactitud_grupo_train
## [1] 0.7922509
## 
## $exactitud_grupo_test
## [1] 0.7903022
\end{verbatim}

\subsubsection{Ejercicio 10}\label{ejercicio-10}

\end{document}
